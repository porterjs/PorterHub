% Helaman.tex
\section{Helaman}
\begin{entry}
\heading{Democracy}
\source{Helaman Ch 1}
\background{Pahoran the Older dies and leaves his three sons (Pahoran, Paanchi, and Pacumeni) to dispute over who becomes the new chief judge over the Nephites.}

\quote{4 Now these are not all the sons of Pahoran (for he had many), but these are they who did contend for the judgment-seat; therefore, \underline{they did cause three divisions among the people.}}

\date{4/5/16}
\thought{It is amazing how just three individiuals were the cause of such division.  We see this kind of division in politics all the time.  Two parties with one or two popular candidates each.  My perspective on this is that different oppinions are important.  We come to better conclusions when we debate ideas.  Even the 12 apostles of the Lord do not always agree on things, but they always are unified.  I think that is the difference between division and contrasting oppinions.}
\question{Can there exist healty division in a political standpoint?}
\keyword{Division}
\keyword{Politics}

\BrText{Pahoran was nominated by democracy.}{
5 Nevertheless, it came to pass that {Pahoran was appointed by the voice of the people} to be chief judge and a governor over the people of Nephi.
}
\BrText{Paanchi accepted that he was losing and joined up with Pahoran and not his other brother Pacumeni.}{
 6 And it came to pass that \textbf{Pacumeni, when he saw that he could not obtain the judgment-seat, he did unite} with the voice of the people.}
 \BrText{Pacumeni does not accept the choice of the people and stirs up trouble.}{
 7 But behold, \textbf{Paanchi, and that part of the people that were desirous that he should be their governor, was exceedingly wroth}; therefore, he was about to flatter away those people to \textbf{rise up in rebellion} against their brethren.}

\end{entry}

%% Chapter Seven %%%%%%%%%%%%%%%%%%%%%%%%%%%%%%%%%%%

%% Chapter Eight %%%%%%%%%%%%%%%%%%%%%%%%%%%%%%%%%%%
\subsection{CH 7-8 Nephi's Prayer of Lamantation}

Nephi comes back from proselyting to the lamanites to find that the Nephites have become wicked and have united with the gadiantan robbers.  He is spotted praying by the people and he confronts them about their iniquities.

\begin{entry}
\heading{Remember Him}
\source{Helaman 7:20}
\quote{20 O, how could you have forgotten your God in the very day that he has delivered you?}
\thought{It is easy to boast in our own success and forget that without the Lord's help we could not have accomplished all that we had.  For example, praying for the Lord's help to succeed in an exam and then forgetting to give thanks for aiding in remembering the material.}
\keyword{Remember}
\keyword{Forget}

\end{entry}


\begin{entry}
\heading{False Security}
\source{Helaman 8:6}

\BrText{Satan's lie of False Security.}{6 And now we know that this is impossible, for behold, we are powerful, and our cities great, therefore our enemies can have no power over us.}

\thought{Believing that you can handle temptation can similarly be destructive.  For instance, the insentive to go to a party outweighing the temptation of alcohol found there in. The same can be said about allowing pornography into your mind and expect to not be effected. }
\keyword{False Security}
\keyword{Pride}

\end{entry}