%Discussion

\section*{Discussion}
\begin{multicols}{2}
The results from this study are very similar to what is accepted as perfect form. The release angle and position match the study that Tran did eight years ago, and the arm movement portion of this study match what is accepted as perfect form in the NBA today.

The motion this study calculated to be the optimal free throw form is close enough to what accepted perfect form looks like to reach a reasonable conclusion that the accepted perfect form is optimal. However, the motions do not match perfectly. Further work is needed to build higher fidelity models to find the true optimality of free throw performance. This model only accounts for the arm motion of a player. Basketball players usually bend their knees when shooting in order to get some upward motion when they release the ball. The model developed for this study does not account for the wrist joint or any forces seen in the wrist or fingers. Another way to move forward with this study is to produce an optimizer that communicates between the two models to improve them both relative to each other.  Implementing this functionality would allow for the objective function to include the repeatability of motions as well as the initial trajectory of the ball upon release.

Even with the limitations of this study, the results are enough to support the current form of free throw shooting as optimal. This study could be used to teach developing athletes more about the physics of shooting a free throw. The fact that optimization techniques agree with basketball experience should make it easy to decide the kind of form an athlete should try to obtain in free throw shooting. However, it is likely predictable that any player that uses tactics less optimal such as the Granny-shot, will have a poor performance given the results of basketball optimization.

The authors hope that the concepts used in this study will be used as a starting point for gaining an understanding of the optimal motions used in sports. The concepts of this study may branch out to other sports which can supply more optimized solutions within athletic sports.

\end{multicols}