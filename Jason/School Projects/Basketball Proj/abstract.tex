\begin{mdframed}[style=MyFrame]
\textbf{ABSTRACT:} 

Arm motion has been diversely studied in the field of biomechanics.  The resources and technology available for studying the motions of the body in sport related activities have been used to benefit athletes.  The role of optimization has been used in the past to improve athletic form.  Computer simulations imitate the mechanics of the body to provide realistic explanations as to why certain motions carry out better performance.  However, optimization can also be used as a tool to understand how a player must practice in order to obtain control and reliability during performance.  This study was comprised of two optimization analyses to determine the best approach to shooting a free throw.  Based on the average height statistics of NBA players in 2015, the optimal path of the arm was determined based on the most reliable trajectory of the ball to score a foul shot. Because of the complexity of the arm, simulating the skeletal and muscular structures of the arm were simplified in a 3-dimensional model.  Muscular activity was incorporated in the design in order to provide an analytical perspective of reliability of a shot contrasted with repeatability. Preliminary results suggested that the traditional free throw shot was likely the most optimal method.  However, further development of the project will incorporate an improved model of the arm and introduce muscular fatigue to better simulate motion repeatability.

    \textbf{KEYWORDS:} Free Throw, Optimization, Arm Motion, Reliability, Traditional, Form, Basketball, Trajectory \\
\end{mdframed}

