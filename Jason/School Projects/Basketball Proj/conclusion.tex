%Conclusion

\begin{center}
\Large Conclusion
\end{center}

\begin{multicols}{2}
The intent of this study was to explore whether or not the currently accepted perfect form for free throw shooting was also the optimal form for free throw shooting. Two optimization problems were solved for this study. The first was a problem based on the kinematics of a basketball from the point of release until it reached the basket. The results agreed with past research, in that the optimal release angle was indeed 52 degrees and the peak height of the basketball was estimated to be approximately 12.4 ft. The second optimization problem that was solved involved finding the optimal motion that would result in a velocity and release angle that was solved for in the first optimization problem. This second problem was solved using an assumption that the optimal motion would be the motion that requires the least change in momentum of the basketball. The results of solving this problem closely agreed with the accepted perfect form. Reasons for small differences in this solution were explored, and further work in this area of study was proposed. The results of this study support that the currently accepted form for free throw shooting is also the optimal way to shoot a free throw.
\end{multicols}