% Introduction



\begin{center}
\Large Introduction
\end{center}


\begin{multicols}{2}
Correctly performing a free throw in the game of basketball requires a complex motion by an athlete. In order to be successful free throw shooters, athletes need to perform their shooting motion in a way that can be repeated consistently many times. This project seeks to mathematically explore whether or not the accepted “perfect form” for free throw shooting is optimal way to shoot a free throw.

The accepted perfect free throw was characterized by Tran in 2008. This study used computer models to get a numerical description of the perfect free throw. For a player that is 6 feet 6 inches tall, a release angle of 52 degrees and a peak height of 12.4 inches. Figure \ref{52degrees} is an illustration that describes the perfect free throw depicted by that study. This study is meant to search free throw form in a broader and simpler manner to either confirm these results or point towards a different form.

\myfigure{L}{\linewidth}{\linewidth}{52degrees}{figure}{Resulting from a study carried out by Tran in 2008, the correct way to shoot a basketball is by releasing it at a $52^o$ angle.}

The most efficient motion path may be in a variety of forms and will be one that takes advantage of the momentum applied in the shooting direction.  For instance, a “Granny-shot” could be a candidate method.  However, the hypothesis of this study was to see if the traditional method of shooting a free throw is the most optimal when considering reliability. 

This study was limited to the motion of one arm performing a free throw shot.  The arm was modeled as a rudimentary skeletal structure of the upper arm and forearm with muscular interaction.  All arm properties were based on relative proportions based on the average height and weight of an NBA basketball player in 2015.  Simplistic kinematic equations were used to derive the shooting path of the ball upon release.  Constraints were enabled to prevent the simulation of joint hyper-extension and muscle overloading.  Optimization was performed on MatLab using a Quasi-Newton Method and a Genetic Algorithm.

The concept of optimizing a free throw shot can be advantageous in a variety of different sports.  If an athlete can understand the physical motions required for optimal performance, he will know exactly what to practice in order to master his sport with greater motion control and consistency. 

\end{multicols}


